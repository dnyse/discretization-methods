\section{Tau Approximation}
We are considering the variable coefficient problem
\begin{equation}
	\frac{\partial u}{\partial t}+\sin (x) \frac{\partial u}{\partial x}=0
	\label{eq:pde3}
\end{equation}
with Dirichlet boundary conditions
\begin{equation}
	u(0,t) = u(\pi, t) = 0.
	\label{eq:bdc3}
\end{equation}
The solution approximation is given by:
\begin{equation}
	u_N(x,t) = \sum_{n=0}^{N+N_b} \hat{u}_n(t)\cos(nx)
	\label{eq:sol_approx}
\end{equation}
where $N_b = 2$ is the number of boundary conditions.\newline
Important to note is that the basis function $\cos(nx)$ does not on its own satisfy the boundary conditions.
The residual is then given as
\begin{equation}
	R_N(x, t) = \frac{\partial u_N}{\partial t}+\sin (x) \frac{\partial u_N}{\partial x}
\end{equation}
Computing each term we get:
\begin{itemize}
	\item \textit{Time derivative}:
	      \begin{equation}
		      \frac{\partial u_N}{\partial t} = \sum_{n=0}^{N+2} \frac{d\hat{u}_n(t)}{dt} \cos(nx)
		      \label{eq:tim_derv3}
	      \end{equation}
	\item \textit{Spatial derivative}:
	      \begin{equation}
		      \frac{\partial u_N}{\partial x} = \sum_{n=0}^{N+2} -\hat{u}_n(t) n \sin(nx)
		      \label{eq:spat_derv3}
	      \end{equation}
\end{itemize}
Substituting both terms back into the residual results in
\begin{equation}
	R_N(x, t) = \sum_{n=0}^{N+2} \frac{d\hat{u}_n(t)}{dt} \cos(nx) -  \sin(x) \sum_{n=0}^{N+2} \hat{u}_n(t) n \sin(nx)
	\label{eq:res3}
\end{equation}
We choose as the test function $\psi_m(x) = \frac{2}{\pi}cos(mx)$ for $m \in [0, N]$ with weight function $w(x) = 1$
\begin{equation}
	(\phi_n, \psi_m)_w = \int_0^{\pi} \cos(mx) \frac{2}{\pi}\cos(nx) dx = \delta_{mn}
	\label{eq:weight_f}
\end{equation}
We require that the residual is orthogonal to these test functions:
$(R_N, \psi_m)_w = 0 \quad \text{for all } m$\\
This gives us:
\begin{equation}
	\left(R_N, \psi_m\right)_w  = \frac{2}{\pi} \int_0^{\pi} \left ( \sum_{n=0}^{N+2} \frac{d\hat{u}_n(t)}{dt} \cos(nx) - \sin(x)  \sum_{n=0}^{N+2} \hat{u}_n(t) n \sin(nx) \right) \cos(mx) dx = 0
	\label{eq:res_der}
\end{equation}
Looking at each term individually
\begin{itemize}
	\item \textit{First term}:
	      \begin{equation}
		      \frac{2}{\pi}\sum_{n=0}^{N+2} \frac{d\hat{u}_n(t)}{dt} \int_0^\pi \cos(nx) \cos(mx) dx
		      \label{eq:ft3}
	      \end{equation}
	      Using the orthogonally property of cosine functions
	      \begin{equation}
		      \int_0^{\pi} \cos(nx)\cos(mx)dx = \begin{cases}
			      \frac{\pi}{2} & \text{if } n = m \neq 0 \\
			      \pi           & \text{if } n = m = 0    \\
			      0             & \text{if } n \neq m
		      \end{cases}
		      \label{eq:case}
	      \end{equation}
	      Resulting in
	      \begin{equation}
		      \frac{2}{\pi}\sum_{n=0}^{N+2} \frac{d\hat{u}_n(t)}{dt}\int_0^{\pi} \cos(nx)\cos(mx)dx = \begin{cases}
			      \frac{d\hat{u}_m(t)}{dt}  & \text{if } m \neq 0 \\
			      2\frac{d\hat{u}_0(t)}{dt} & \text{if } m = 0
		      \end{cases}
		      \label{eq:cases2}
	      \end{equation}
	\item \textit{Second Term}:
	      \begin{equation}
		      -\frac{2}{\pi}\sum_{n=0}^{N+2} \hat{u}_n(t) n \int_0^\pi \sin(x) \sin(nx) \cos(mx) dx
		      \label{eq:sterm}
	      \end{equation}
	      We can rewrite the integral as follows:
	      \begin{equation}
		      \int_0^\pi \sin(x) \sin(nx) \cos(mx) dx = \frac{1}{2}\int_0^{\pi} \cos((n-1)x) - \cos((n+1)x) \cos(mx) dx
		      \label{eq:oou}
	      \end{equation}
	      splitting up this integral leads to
	      \begin{equation}
		      \begin{aligned}
			      \int_0^{\pi} \cos((n-1)x)\cos(mx)dx & = \frac{1}{2}\int_0^{\pi} [\cos((n-1+m)x) + \cos((n-1-m)x)]dx \\
			      \int_0^{\pi} \cos((n+1)x)\cos(mx)dx & = \frac{1}{2}\int_0^{\pi} [\cos((n+1+m)x) + \cos((n+1-m)x)]dx
		      \end{aligned}
		      \label{eq:int}
	      \end{equation}
	      For a cosine integral:
	      \begin{equation}
		      \int_0^{\pi} \cos(kx)dx = \begin{cases}
			      \pi                      & \text{if } k = 0    \\
			      \frac{\sin(k\pi)}{k} = 0 & \text{if } k \neq 0
		      \end{cases}
		      \label{eq:cos}
	      \end{equation}
	      Therefore these integrals are non-zero only when:
	      \begin{equation}
		      \begin{aligned}
			      (n - 1 + m) & = 0 \Rightarrow n = 1 - m                                        \\
			      (n - 1 - m) & = 0 \Rightarrow n = m + 1                                        \\
			      (n + 1 + m) & = 0 \Rightarrow n = - (m+1) \text{ (impossible for } n,m \geq 0) \\
			      (n + 1 - m) & = 0 \Rightarrow n = m - 1                                        \\
		      \end{aligned}
		      \label{eq:caso}
	      \end{equation}
	      Therfore for $m \geq 1$ the only non-zero contributions come from
	      \begin{itemize}
		      \item $n=m+1$: $\frac{m+1}{2}\hat{u}_{m+1}$
		      \item $n=m-1$: $\frac{m-1}{2}\hat{u}_{m-1}$
	      \end{itemize}
	      For $m=0$ the non-zero term is
	      \begin{itemize}
		      \item $n=1$: $-\frac{1}{2}\hat{u}_1$
	      \end{itemize}
\end{itemize}
Combining both terms for $m \geq 1$ we end up with
\begin{equation}
	\frac{\hat{u}_m (t)}{dt} = \frac{1}{2} \left [ (m+1) \hat{u}_{m+1}(t) - (m-1)\hat{u}_{m-1}(t)\right]
	\label{eq:sol3a}
\end{equation}
and for $m=0$
\begin{equation}
	\frac{\hat{u}_0 (t)}{dt} = \frac{1}{4}\hat{u}_1(t)
	\label{eq:sol3b}
\end{equation}
Applying the boundary conditions then results in
\begin{enumerate}
	\item $u_N(0,t) = \sum_{n=0}^{N+2} \hat{u}_n (t) \cos(0) =  \sum_{n=0}^{N+2} \hat{u}_n (t) = 0$
	\item $u_N(\pi,t) = \sum_{n=0}^{N+2} \hat{u}_n (t) \cos(m\pi) =  \sum_{n=0}^{N+2} (-1)^n \hat{u}_n (t) = 0$
\end{enumerate}
which provide additional constraints and the initial conditions are for $m\in [0,$
\begin{equation}
	\hat{u}_m(0) = \frac{2}{\pi}\int_0^{\pi} g(x)\cos(mx)dx
	\label{eq:init3}
\end{equation}
