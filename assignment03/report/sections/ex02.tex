\section{Fourier-Galerkin Approximation: Exercise 2}
We are considering the variable coefficient problem
\begin{equation}
	\frac{\partial u}{\partial t}+\sin (x) \frac{\partial u}{\partial x}=0
	\label{eq:pde2}
\end{equation}
with Dirichlet boundary conditions
\begin{equation}
	u(0,t) = u(\pi, t) = 0
	\label{eq:bdc2}
\end{equation}
For our Basis function we choose the sine as our basis due to the it naturally satisfying the boundary condition
\begin{equation}
	\sin(nx) |_{x=0} = \sin(nx) |_{x=\pi} = 0
	\label{eq:sin_bdc}
\end{equation}
therefore we define for the approximation
\begin{equation}
	u_N(x, t) = \sum_{n=1}^{N} \hat{u}_n(t) \sin(nx)
	\label{eq:uN2}
\end{equation}
If we now follow the Galerking Approach first we substitute our approximation $u_N$ into the PDE:
\begin{equation}
	\frac{\partial u_N}{\partial t}+\sin (x) \frac{\partial u_N}{\partial x}=0
	\label{eq:un_pde2}
\end{equation}
The residual is then given as
\begin{equation}
	R_N(x, t) = \frac{\partial u_N}{\partial t}+\sin (x) \frac{\partial u_N}{\partial x}
\end{equation}
Computing each term we get:
\begin{itemize}
	\item \textit{Time derivative}: 
		\begin{equation}
			\frac{\partial u_N}{\partial t} = \sum_{n=1}^{N} \frac{d\hat{u}_n(t)}{dt} \sin(nx)
			\label{eq:tim_derv2}
		\end{equation}
	\item \textit{Spatial derivative}:
\begin{equation}
	\frac{\partial u_N}{\partial x} = \sum_{n=1}^{N} \hat{u}_n(t) n \cos(nx)
			\label{eq:spat_derv2}
		\end{equation}
\end{itemize}
Therefore the residual becomes
\begin{equation}
	R_N(x, t) =  \sum_{n=1}^{N} \frac{d\hat{u}_n(t)}{dt} \sin(nx) + \sin(x)  \sum_{n=1}^{N} \hat{u}_n(t) n \cos(nx)
\end{equation}
For the final step we want to make the residual orthogonal to the basis function. Hence for sine functions on the interval $[0,\pi]$ with weight function $w(x)=1$, we have:
\begin{equation}
	(\psi_n, \psi_m)_w = \int_0^{\pi} \sin(mx)\sin(nx) dx = \frac{\pi}{2}\delta_{mn}
	\label{eq:weight_f}
\end{equation}
Therefore, $\gamma_m = \frac{\pi}{2}$ and our test functions should be:
\begin{equation}
\psi_m = \frac{2}{\pi}\sin(mx)
	\label{eq:test2}
\end{equation}
We require that the residual is orthogonal to these test functions:
$(R_N, \psi_m)_w = 0 \quad \text{for all } m \in[1, N]$
This gives us:
\begin{equation}
	\left(R_N, \psi_m\right)_w  = \frac{2}{\pi} \int_0^{\pi} \left ( \sum_{n=1}^{N} \frac{d\hat{u}_n(t)}{dt} \sin(nx) + \sin(x)  \sum_{n=1}^{N} \hat{u}_n(t) n \cos(nx) \right) \sin(mx) dx = 0
	\label{eq:res_der}
\end{equation}
Breaking this into two terms:
Term 1:
$$\frac{2}{\pi}\int_0^{\pi} \sum_{n=1}^{N} \frac{d\hat{u}_n(t)}{dt} \sin(nx)\sin(mx) dx$$
Using the orthogonality property:
$$\frac{2}{\pi} \cdot \frac{\pi}{2} \frac{d\hat{u}_m(t)}{dt} = \frac{d\hat{u}_m(t)}{dt}$$
Term 2:
$$\frac{2}{\pi}\int_0^{\pi} \sin(x)\sum_{n=1}^{N} \hat{u}_n(t) n\cos(nx)\sin(mx) dx$$
We can rewrite this as:
$$\frac{2}{\pi}\sum_{n=1}^{N} \hat{u}_n(t) n \int_0^{\pi} \sin(x)\cos(nx)\sin(mx) dx$$
To evaluate the integral, we use the trigonometric identity:
$$\sin(a)\cos(b) = \frac{1}{2}[\sin(a+b) + \sin(a-b)]$$
