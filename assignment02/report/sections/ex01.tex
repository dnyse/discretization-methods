\section{ODD vs. EVEN Method}

Based on the numerical evidence presented in Table~\ref{tab:fourier-comparison}, the EVEN Fourier differentiation method is generally more accurate and computationally efficient than the ODD method for approximating derivatives of the function $u(x) = \exp(k\sin(x))$. The EVEN method achieves smaller maximum relative errors in most cases while requiring fewer grid points, which translates to computational savings.
The superior performance of the EVEN method is particularly pronounced for higher values of $k$ where the function oscillates more rapidly and presents a more challenging differentiation problem. This suggests that the EVEN method may be particularly advantageous for highly oscillatory functions.
\begin{table}[H]
	\centering
	\begin{tabular}{|c|cc|cc|}
		\hline
		\multirow{2}{*}{$k$} & \multicolumn{2}{c|}{ODD Method} & \multicolumn{2}{c|}{EVEN Method}                                       \\
		\cline{2-5}
		                     & Minimum $N$                     & Max Relative Error               & Minimum $N$ & Max Relative Error    \\
		\hline
		2                    & 22                              & $3.78 \times 10^{-6}$            & 20          & $3.13 \times 10^{-6}$ \\
		4                    & 32                              & $5.62 \times 10^{-6}$            & 32          & $3.02 \times 10^{-7}$ \\
		6                    & 42                              & $4.12 \times 10^{-6}$            & 40          & $1.19 \times 10^{-6}$ \\
		8                    & 52                              & $2.43 \times 10^{-6}$            & 48          & $3.37 \times 10^{-6}$ \\
		10                   & 60                              & $7.89 \times 10^{-6}$            & 56          & $8.22 \times 10^{-6}$ \\
		12                   & 70                              & $3.87 \times 10^{-6}$            & 68          & $5.79 \times 10^{-7}$ \\
		\hline
	\end{tabular}
	\caption{Comparison of ODD and EVEN Fourier Differentiation Methods for $u(x) = \exp(k\sin(x))$}
	\label{tab:fourier-comparison}
\end{table}
